\documentclass [12pt,english]{article}

\setlength{\oddsidemargin}{0in}
\setlength{\evensidemargin}{0in}
\setlength{\textwidth}{6.5in}
\setlength{\textheight}{8in}
\setlength{\topmargin}{0in}
\setlength{\columnsep}{.25in}

\usepackage{babel,amsmath,epsfig}
\usepackage{url}

\renewcommand{\baselinestretch}{1.5}
\newcommand{\eqn}[1]{(\ref{eqn:#1})}
\newcommand{\lab}[1]{\label{eqn:#1}}
\def\eps{\epsilon} 
\def\beq{\begin{equation}}   
\def\eeq{\end{equation}}
\def\ub{\bf u}
\def\delt{\nabla^2}  \def\div{\nabla \cdot} \def\grad{\nabla} 
\def\pp#1#2{ {\partial #1 \over \partial #2} } 
\def\frac#1#2{ {#1\over#2}}
\def\fracP#1#2{{{\partial#1}\over{\partial#2}}}
\def\faf#1{\tilde{f_{#1}}}
\def\saf#1{\tilde{s_{#1}}}
\def\LF#1{\label{Fig.#1}}
\def\RF#1{figure~\ref{Fig.#1}}
\def\RFS#1#2{figures.\ref{Fig.#1}\ref{Fig.#2}}
\def\bg#1{\mbox{\boldmath $#1$}}
\def\wid{b}

\hyphenation{geo-stro-phic}
\hyphenation{semi-geo-stro-phic}
\hyphenation{wa-ve-num-bers}
\hyphenation{two-di-men-sio-nal}


\begin{document}

\title{\Large{\bf Fluid Dynamics --- Numerical Techniques}\\
{\bf MATH5453M Numerical Exercise 3, 2016}\\
{\small\bf Due date: December 9$^{\rm th}$ 2016}}
\author{}
\date{}

\maketitle

\vspace*{-2cm}\noindent
{\em Keywords: diffusion equation,
finite element method for linear and nonlinear diffusion equations,
forward Euler \& Crank-Nicolson--scheme, groundwater modelling}. 
Sources: Lecture Notes, Van der Kan et al. 2005, Morton and Mayers (2005), Internet.

\section{Ground water model}

Consider the width-averaged nonlinear diffusion equation modelling groundwater flow in a channel (Barenblatt 1996)
\begin{align}
\partial_t(w_v h_m)-\alpha g\partial_y(w_v h_m\partial_y h_m) =& {w_v R}/{(m_{por} \sigma_e)}\label{eqngroundw}
\end{align}
with groundwater level and variable $h_m=h_m(y,t)$ [$L$] above a horizontal datum $z=0$,
channel width $w_v\approx 0.1{\rm m}$, acceleration of gravity $g=9.81{\rm m}/{\rm s}^2$ [$L/T^2$], derivative $\partial_y$
in the along-channel direction $y\in[0,L_y]$ of a cell of length $L_y\approx 0.85{\rm m}$ [$L$], porosity $m_{por}\in[0.1,0.3]$,
given rainfall $R=R(t)$ [$L/T$], the fraction of a pore $\sigma_e\in[0.5,1]$ that can be filled with water due to residual air, factor
\begin{align}
\alpha =& {k}/{(\nu m_{por}\sigma_e)}
\end{align}
with permeability $k\in[10^{-8},10^{-9}]{\rm m}^2$ [$L^2$] and viscosity of water $\nu=10^{-6}{\rm m}^2/{\rm s}$ [$L^2/T$].
Boundary conditions are no flow at $y=L_y$ such that
\begin{align}
\partial_y h_m =& 0
\end{align}
and a Dirichlet condition at $y=0$, the other channel end, equal to the water level $h_{cm}(t)$ in a short outflow canal
\begin{align}
h_m(0,t) = h_{cm}(t).
\end{align}
The initial condition is $h_m(y,0)=h_{m0}(y)$. Make a sketch of the situation. % is given in Fig.~\RF{sketgrw}.

The assumptions are that the groundwater level stays underground thus not inducing any surface run-off and that the
flow is hydrostatic, with variations in the horizontal $y$--direction much longer than the vertical length scales.
The canal level $h_{cm}(t)$ holds in a short channel of length $L_c$ between $-L_c<y<0$ with a weir at $y=-L_c$.
The water level at the weir is critical, meaning that the flow speed $V_{c}=\sqrt{g h_c}$ is critical at $y=-L_c$.
Assuming stationarity and by using Bernoulli's equation to link the speed and water depth $\{V_{cm},h_{cm}\}$ in the channel to that at the weir $\{V_c,h_c\}$ one obtains that
\begin{align}
g h_{cm}+\frac 12 V_{vm}^2 \approx g h_{cm} = g h_c+ \frac 12 V_c^2 =& \frac 32 g h_c,\\
\Longrightarrow h_c = \frac 23 h_{cm}\quad\textrm{such~that}\quad Q_c = h_c V_c =& \sqrt{g}\max{(2 h_{cm}/3,0)}^{3/2},
\end{align}
assuming in addition that $V_{cm}^2\ll g h_{cm}\approx 0$ (see, e.g., Munson et al. 2005). 

To obtain further insight, we rewrite and analyse \eqref{eqngroundw} next.
The groundwater equation is clearly a continuity equation
\begin{align}
\partial_t(w_v h_m) +\partial_y(v h_m) =& \frac {w_v R}{m_{por}\sigma_e}
\end{align}
with the Darcy velocity
\begin{align}
  v =& -\frac {k g}{\nu m_{por} \sigma_e}\partial_y h_m  
\end{align}
and Darcy flux
\begin{align}
  Q \equiv w_v q \equiv &  w_v h_m v = -w_v h_m \frac {\kappa}{\mu}\partial_y p = -w_v h_m \frac {\kappa}{\nu\rho_0}\partial_y p\nonumber\\
  \approx &  - \frac {\kappa g}{\nu}\partial_y h_m =-w_v h_m\frac {k g}{\nu m_{por}\sigma_e} \partial_y h_m
  = -w_v \alpha g \partial_y\left(h_m^2 /2\right) 
\end{align}
with density of water $\rho_0$ and where we used the hydrostatic approximation and depth-integration
to the free surface at $z=h_m$ by using $\partial_y p/\rho_0 \approx g\partial_y h_m$.
The first and last term in \eqref{eqngroundw} display the water balance as follows, in the case that there is no $y$--dependence:
\begin{align}
\partial_t h_m =& {R}/{(m_{por}\sigma_e)}.
\end{align}
Hence, for $h_m$ zero initially and constant rainfall, we find $h_m = t R /(m_{por}\sigma_e)$
showing that for $m_{por}=\sigma_e=1$ unity the groundwater level rises directly with rainfall,
while it rises faster for general $m_{por}<1$ and $\sigma_e<1$, showing that the modelling of rainfall supply is consistent.
Hence, the canal level is modelled by the outflow at $y=-L_c$ and inflow at $y=0$ as follows
\begin{align}
  L_c w_v \frac {{\rm d}h_{cm}}{{\rm d} t} =& m_{por}\sigma_e Q_0-Q_c \equiv w_v m_{por}\sigma_e\frac 12\alpha g \partial_y(h_m^2)|_{y=0} -w_v \sqrt{g} \max{\left(\frac 23 h_{cm}(t),0\right)}^{3/2}.\nonumber
  \end{align}

In summary, the complete mathematical groundwater model is:
\begin{align}
  \partial_t(w_v h_m)-\alpha g\partial_y(w_v h_m\partial_y h_m) =& \frac {w_v R}{m_{por} \sigma_e}\quad\textrm{in}\quad
y\in[0,L_y]\label{eqngroundwsys}\\
  \partial_y h_m =& 0\quad\textrm{at}\quad y=L_y\\
  h_m(0,t) =& h_{cm}(t)\quad\textrm{at}\quad y=0\\
  L_c w_v \frac {{\rm d}h_{cm}}{{\rm d} t} =&
w_v m_{por}\frac {\sigma_e}{2}\alpha g \partial_y(h_m^2)|_{y=0}
-w_v \sqrt{g} \max{\left(\frac 23 h_{cm}(t),0\right)}^{3/2}\label{eqngroundwsysc}
\end{align}
plus initial conditions for $h_m$ and $h_{cm}$.
The multiplication by $m_{por}\sigma$ accommodates the flow out of the groundwater matrix into open space (?),
while the $y$--derivative of $h_m^2$ has been taken rather than using  $h_m \partial_y h_m$ as
otherwise a simple explicit discretisation with
$h_m(0,0)=0$ is and $h_{cm}(0)=0$ will not lead to water flux into the canal. 
It may be necessary to rewrite the equations such that they fit an appropriate finite element weak formulation.

\section{Questions}

\begin{enumerate}

\item a) Write your own numerical program solving the heat equation $u_t=u_{xx}$ with Dirichlet boundary data as well as initial data using
finite elements and using both the forward Euler and Crank-Nicolson time stepping scheme.
You can, e.g., use the Thomas algorithm or the backslash in matlab.
Provide all steps required in the finite element discretisation in detail on paper before you start programming.

b) Argue how you can use the stability results for the finite difference scheme to find a time step for the finite element scheme.
Van de Kan et al. (2005) provide more information on how to find the actual finite element stability criterion.

c) Compare the numerical results of the two implementations, for $\theta=0,1/2$.
Use, for example, a top-hat profile with unit value in the middle and zero values at the edges, and
$\exp(-\alpha(x-x_m)^2)$ for sufficiently large $\alpha$ and $x_m=1/2$ lying within your domain.
Compare your finite difference and finite element solutions.
{\em Advanced: make a table of convergence using the $L^{\infty}$--error.}

d) Numerically investigate the stability for $\theta=0,1/2$.
Demonstrate this by succinctly showing your results in appropriate graphs.
E.g., reproduce a relevant figure in Chapter 2 of Morton and Mayers (2005).

\item Discretise  \eqref{eqngroundwsys}--\eqref{eqngroundwsysc} using an explicit finite element space and time discretisation
that keeps the adjoint structure in tact.
Provide all steps required in detail first. Use an explicit scheme.
Use the time step criterion for the finite difference case to obtain and state a time step estimate.
Show how the flux at $y=0$ arising after multiplication of the main partial differential equation by a test function and
subsequent integration by parts can be eliminated. Should one take $h_m(0,t)=h_{cm}(t)$ in this calculation?

\item Perform numerical simulations for the parameter values:
\begin{align}
  m_{por} =&  0.3,\quad \sigma_e = 0.8,\quad
  L_y = 0.85{\rm m}\quad k = 10^{-8}{\rm m}^2,\quad w_v = 0.1{\rm m},\\ % \quad
  R_{max} =& 0.000125{\rm m}/{\rm s},\quad L_c=0.05{\rm m}.
\end{align}
Start with $h_m(y,0)=0, h_{cm}(0)=0$.
Model the system for $t=0,\dots,100$s and give output profiles of $h_{cm}, R(t)$ every $2$s and $h_m$ every $10$s.
Demonstrate numerical convergence of your scheme for the solution at $t=100$s.
Has the system reached steady state; what is the steady-state value of $h_{cm}$ and steady-state profile $h_{m}(y)$?
Finally, vary the rain every $10$s, apply rain $1,2,4$ or $9$s out of $10$s or at fixed lower percentages for fixed $R_{max}$ 
and display the changes in $h_m(y,t), h_{cm}(t), R(t)$ sensibly.
Compare with your finite difference solver and interpret your results.
{\em Hint:} Perhaps first implement the easier problem with $h_{cm}=0.07{\rm m}$ fixed and then add the canal equation. 

\item {\em Advanced:} Use a Crank-Nicolson scheme instead to solve the above problem (in Firedrake). Detail your discretisation.
Find a method to solve this nonlinear algebraic system (e.g., use Picard or Newton iteration).
Implement it, demonstrate that the iteration converges and then
compare it with the explicit (finite element and difference) discretisation.
  

\end{enumerate}


\centerline{\bf References}

\noindent
Barenblatt, G.I. 1996: {\em Scaling, self-similarity, and intermediate asymptotics.} CUP. % Cambridge University Press.

\noindent
Bokhove, O., Zweers, W. 2016: Wetropolis design including groundwater cell. Photos and videos found at:
{\em Resurging Flows} facebook page \url{https://www.facebook.com/resurging.flows}
Presentation: \url{http://www1.maths.leeds.ac.uk/mathsforesees/projects.html}

\noindent
Van de Kan, J., Segal, A., van der Molen, F. (2005)
{\em Numerical Methods in Scientific Computing}, VSSD, 279~pp.

\noindent
K.W. Morton and D.F. Mayers 2005:
{\em Numerical solution of partial differential equations.} Cambridge University Press, 278~pp.

\noindent
B.R. Munson, D.F. Young and T.H. Okiishi 2005
{\em Fundamentals of fluid mechanics}. Wiley. 

\noindent
Internet, e.g., for table of convergence, $L^{\infty}$--error, et cetera.

%\end{document}

\newpage

\appendix

\section{Solutions}

Given Dirichlet boundary conditions $u(0,t)=u(L,t)=0$, multiply
\begin{align}
u_t=u_{xx}
\end{align}
with a test function $v=v(x)$ with $v(0)=v(L)=0$ and integrate (by parts) to obtain
\begin{align}
 \int_0^L v u_t\,{\rm d} x = -\int_0^{L} v_x u_{x}\,{\rm d} x,
\end{align}
in which the boundary terms cancel because $v(0)=v(L)=0$.
This should be the weak formulation for Firedrake using $v$ and $u$ as a continuous Galerkin approximation
with Dirichlet boundary conditions.
For $L=1$, the initial conditions could be, e.g.,
$u(x,0)= A\,e^{-\alpha|x-x_m|^2}$ with $x_m=L/2$ and suitable $\alpha$ such that $u(0,t)=u(L,t)\approx 0$;
$u(x,0)=A\,x\,(1-x)$; or, $u(x,0)= A-\alpha(x-x_m)^2$ for $|x-x_m| \le \sqrt{A/\alpha}$.
Using compact (linear) basis functions $v=\varphi_{i'}(x)$ with $i'=2,\dots,N$ and Galerkin expansions
\begin{align}
u(x,t)\approx u_h(x,t) =&  u_j(t)\varphi_j(x) = u_1\varphi_1+u_{N+1}\varphi_{N+1}+u_{j'}\varphi_{j'}
= u_{j'}\varphi_{j'}, 
\end{align}
since $u_1=u_{N+1}=0$, we obtain
\begin{align}
 M_{i' j } \frac {{\rm d}u_{j}}{{\rm d}t} =& -S_{i' j} u_j
 \Longrightarrow M_{i' j } \frac {{\rm d}u_{j'}}{{\rm d}t} = -S_{i' j'} u_{j'}\\
 M_{i' j'} =& \int_0^L\varphi_{i'}\varphi_{j'} \,{\rm d} x\quad\textrm{and}\quad
 S_{i' j'} = \int_0^L\partial_x \varphi_{i'}\partial_x\varphi_{j'} \,{\rm d} x,
\end{align}
with $(N-1)\times(N-1)$--matrices $M_{i' j'}$ and $S_{i'j'}$.
Using the forward Euler and Crank-Nicolson time integration schemes,
the weak Firedrake form and the matrix formulations become
\begin{align}
  \int_0^L v u^{n+1}\,{\rm d} x = \int_0^{L} v u^n-\Delta t v_x u^n_{x}\,{\rm d} x\\
  \int_0^L v u^{n+1}+\frac 12 \Delta t \int_0^{L} v_x u^{n+1}_{x}\,{\rm d} x = \int_0^L v u^{n}-\frac 12 \Delta t \int_0^{L} v_x u^{n}_{x}\,{\rm d} x\\
  M_{i' j' }  {u_{j'}}^{n+1} =& M_{i' j'} u_{j'}^{n}-\Delta t S_{i' j'} u_{j'}^n\\
  \bigl( M_{i' j } +\frac 12 \Delta t S_{i' j'}\bigr) {u_{j'}}^{n+1}
    =& \bigl(M_{i' j'} -\frac 12 \Delta t S_{i' j'}\bigr)u_{j'}^n,
\end{align}
for which the initial condition needs to be projected onto the finite element basis using $\int_0^L v u_h(x,0)\,{\rm d}x=\int_0^Lv u(x,0)\,{\rm d} x$
yielding $u_{j'}(0) = M_{i'j'}^{-1}\int_0^L \varphi_{i'}u(x,0)\,{\rm d} x$.

\vskip 12pt\noindent
{\em Groundwater model FEM:}
Multiplying \eqref{eqngroundwsys} times test function $q=q(y)$ with $h_m(0,t)=h_{cm}(t)$ yields, after integration by parts 
and using that $\partial_y h_m=0$ at $y=L_y$ as well as $h_m(0,t)=h_{cm}(t)$, that
\begin{align}
\int_0^{L_y}  q \partial_t h_m+\alpha g h_m \partial_y q \partial_y h_m\, dy
+\underline{\alpha g q(0) h_m\partial_y h_m|_{y=0}} 
=& \int_0^{L_y} \frac {q R}{m_{por} \sigma_e} \, dy\\
L_c \frac {{\rm d}h_{cm}}{{\rm d} t}= m_{por}\sigma_e\underline{\alpha g \frac 12 \partial_y(h_m^2)|_{y=0}}- &
\sqrt{g} \max{\left(\frac 23 h_{cm},0\right)}^{3/2},\\
\underline{\alpha g \frac 12 \partial_y(h_m^2)|_{y=0}} = L_c \frac {1}{m_{por}\sigma_e}\frac {{\rm d}h_{cm}}{{\rm d} t}
+ & \frac {\sqrt{g}}{m_{por}\sigma_e} \max{\left(\frac 23 h_{cm},0\right)}^{3/2},
\end{align}
in which the underlined $\partial_y(h_m^2)|_{y=0}$ is eliminated between the two equations.
These two equations are thus combined to obtain
\begin{align}
\int_0^{L_y}  q \partial_t h_m\, d y+\frac {q(0)  L_c}{m_{por}\sigma_e} \frac {{\rm d}h_{cm}}{{\rm d} t}
=& \int_0^{L_y} -\alpha g h_m \partial_y q \partial_y h_m+\frac {q R}{m_{por} \sigma_e} \,dy \nonumber\\
 & -\frac {q(0)}{m_{por}\sigma_e} \sqrt{g} \max{\left(\frac 23 h_{cm},0\right)}^{3/2}\\
& \textrm{or}\nonumber\\
\int_0^{L_y}  q \partial_t h_m\, d y+\frac {q(0)  L_c}{m_{por}\sigma_e} \frac {{\rm d}h_{m}(0,t)}{{\rm d} t}
=& \int_0^{L_y} -\alpha g h_m \partial_y q \partial_y h_m+\frac {q R}{m_{por} \sigma_e} \,dy \nonumber\\
 & -\frac {q(0)}{m_{por}\sigma_e} \sqrt{g} \max{\left(\frac 23 h_{m}(0,t),0\right)}^{3/2}
\end{align}
Note that $q$ remains unconstraint.
Consider piecewise linear finite elements.
The forward Euler and Crank-Nicolson time discretizations yield the (Firedrake) weak formulations
\begin{align}
&\int_0^{L_y}  q h_m^{n+1}\, d y +\frac {L_c h_{cm}^{n+1}}{m_{por}\sigma_e} = \int_0^{L_y} q h_m^{n}\, dy+\frac {L_c h_{cm}^{n}}{m_{por}\sigma_e}\nonumber\\
&\qquad +\Delta t \int_0^{L_y} \left(-\alpha g h_m^n \, \partial_y q \partial_y h_m^n+\frac {q R^n}{m_{por} \sigma_e}\right)\, dy
-\Delta t \frac {\sqrt{g}}{m_{por}\sigma_e} \max{\left(\frac 23 h^n_{cm},0\right)}^{3/2} \\
%
%
& \int_0^{L_y}  q h_m^{n+1}\, d y +\frac {L_c h_{cm}^{n+1}}{m_{por}\sigma_e}
+\frac 12\Delta t \int_0^{L_y}\alpha g h_m^{n+1} \,\partial_y q \partial_y h_m^{n+1}\, dy
+\frac 12\Delta t \frac {\sqrt{g}}{m_{por}\sigma_e} \max{\left(\frac 23 h^{n+1}_{cm},0\right)}^{3/2}
\nonumber\\
&\quad = \int_0^{L_y} q h_m^{n}\, dy+\frac {L_c h_{cm}^{n}}{m_{por}\sigma_e}
+\frac 12\Delta t\int_0^{L_y}\left(-\alpha g h_m^n \partial_y q \partial_y h_m^n + \frac {q (R^n+R^{n+1})}{m_{por} \sigma_e}
\right)\, dy\nonumber\\
&\qquad\qquad -\frac 12\Delta t \frac {\sqrt{g}}{m_{por}\sigma_e} \max{\left(\frac 23 h^n_{cm},0\right)}^{3/2}.
\end{align}
Taking expansions $q=\varphi_i(x)$ and $h_m=h_j\varphi_j(x)$, for all $i,j=1,\dots,N_n$ with $N_n$ nodes, as well as $h_1=h_{cm}$
the matrix form for the forward Euler case becomes
\begin{align}
& M_{ij} h_{j}^{n+1}+\frac {L_c h_{1}^{n+1}}{m_{por}\sigma_e}\delta_{i1} = M_{ij} h_j^{n}+\frac {L_c h_{1}^{n}}{m_{por}\sigma_e}\delta_{i1}+\Delta t b_{i}^n
  -\Delta t \frac {\sqrt{g}}{m_{por}\sigma_e} \max{\left(\frac 23 h^n_{1},0\right)}^{3/2}\delta_{i1}\nonumber\\
  & \qquad\qquad \textrm{for}\quad i=1,\dots,N_n\\
M_{ij} = & \int_0^{L_y} \varphi_i\varphi_j  \, dy\quad\textrm{and}\quad
b_{i}^n = \int_0^{L_y} \left(-\alpha g h_m^n \, \partial_y \varphi_{i} \partial_y h_m^n
+\frac {\varphi_i R^n}{m_{por} \sigma_e}\right)\, dy
\end{align}
with the Einstein summation convention used (here for $j$) and Kronecker delta symbol $\delta_{i1}=1$ when $i=1$ and $\delta_{i1}=0$ when $i\ne 0$.
Note that for a piecewise linear finite element approximation the unknown vector
$$(h_{cm}^{n+1},h_2^{n+1},\dots,h_j^{n+1},\dots,h_{N_n}^{n+1})^T$$
includes the moor variables as well as the canal variable combined.

This formulation yields (visually) the same answer as the finite difference calculations!

\end{document}

\newpage

\appendix

\section{How to write a computer program}

Writing a computer program, e.g., in Matlab, Python or any other programming language,
consists typically of the following steps. We will use writing a code for the explicit discretization
of the diffusion equation as example.

Hence, these steps are:
\begin{itemize}
\item Derive and write out the mathematics of the complete numerical algorithm, e.g., including
discrete boundary and initial conditions, the mesh and its numbering, the
equation updates for each unknown, using index notation or loops.
\item Having completed this mathematical part, write a pseudo-code in words and/or block format;
these blocks can in principle also become subroutines or classes or functions
as this will improve the organization of the final program:
\begin{description}
\item{-} Introduce comments, starting with the title and goal of the program;
\item{-} Define all parameters and variables required, clearly distinguishing these two, e.g.:
$L, dx, Nx, Nt, dt, Tend, tmeasure, dtmeasure,mu=fac*dt/(dx*dx),tijd$
(domain length, spatial step, number of grid cells, time step, final time, measurement time and increment, factor $fac$ for stability, time variable $tijd$),
et cetera, and the arrays $u=u[j], unew[j]$ for $j=1,2,3,\dots,Nx+1$ et cetera;
\item{-} Initialize some variables and plot the initial condition.
\item{-} Create a for or while loop over time.
\item{-} Create a for or while loop over space, or use the vector form in matlab, e.g.:\\
$unew[2:Nx]= u[2:Nx]+mu*(u[1:Nx-1]-2*u[2:Nx]+u[3:Nx+1])$
\item{-} Within these loops arange to plot at appropriate times.
\end{description}
Note that it is not a good idea to store $U^n_j$ as a matrix array as it will lead to an overflow of data.
So do not store all the space--time data.
\item The next step is to write a piece of code along these lines:
\begin{description}
\item{-} Test your code after every small stage/block by running it; deal with error messages, e.g.,
explicitly check the range of your arrays, et cetera;
\item{-} Initially plot after every time step, plot the new updates of the variables,
and just run your code for a few time steps to see whether matters behave appropriately or blow up;
\item{-} Verify your code against exact solutions, other people's solutions,
solutions using a different method/program (as in our exercise),
graphs in books, and/or high-resolution runs.
\end{description}
\end{itemize}


%\newpage
\appendix

\section{Remarks and/or common mistakes}

%\subsection{Table of Convergence}

A table of convergence can be used to determine and/or confirm the formal
order of convergence of your numerical scheme by simulations.
Consider the case with a variable $u(x,t)$ and initial condition $u_0(x,0)$,
and simulate to a final time $t=T$ with a fixed spatial grid of size $\Delta x$ and (fixed) time step $\Delta t$.
The easiest thing to do is to pick a time $t_c<T$ at which the solution has sufficient spatial structure
when one checks the convergence in space and still has sufficient temporal structure when (also) investigating
the convergence in time.
Denote the numerical solution at this time $t_c$:
by $U_j$ or $U_j^{n_c}$ and the exact or a very high resolution solution by $u(x,t_c)$.
Here $x_j=(j-1)\Delta x$ is the coarse grid position one starts with $j=1$.
Denote the fine grid positions by $x_{j'}=(j'-1)\Delta x'$,
i.e., we used two different meshes with two different mesh sizes.

Remarks:
\begin{itemize}
\item Choosing a final decayed state with $u(x,t_c=T)\approx 0$ is therefore not appropriate.
\item Define the norms you are using clearly and in mathematical terms, not as
some internal function in some programming language.
That is the $L^{\infty}$--error is:
\begin{align}
L_{\infty} = max_{j}(|U_j^{n_c}-u(x_j,t_c)|),
\end{align}
which is trivial to compute by looping over all grid points.
In case you only have a refined solution, either take care that the fine grid also includes the course grid position
$x_j$ or interpolate the solution between the appropriate positions $x_{j'}\le x_j\le x_{j'+1}$ (noting that $j$ and $j'$ concern
indices on different grids).
Make a ``continuum'' numerical solution denoted by $U(x,t)$ (e.g., by interpolation).
The $L_2$--error is, using indexing $j=1,\dots,N_x+1$ and a numerical integral approximation
for the integral:
\begin{align}
L_{2}=& \sqrt{\frac 1L \int_0^L \,(U(x,t_c)-u(x,t_c))^2{\rm d}x}\nonumber\\
\approx & \frac {1}{\sqrt{L}}\sqrt{\bigl(\frac 12(U_0-u(0,t_c))^2+\frac 12(U_{N_x}-u(L,t_c))^2+\sum_{j=2}^{N}(U_j^{n_c}-u(x_j,t_c))^2\bigr)\Delta x}.
\end{align}
Even a lousy integral approximation tends to suffice to demonstrate convergence. i.e.,
it is usually not worth spending time on accurate integration routines to determine these norms.
\item When a numerical method is second order, the error scales like $(\Delta x)^2$, i.e.
\begin{align}
Error = C (\Delta x)^2
\end{align}
for some constant $C$ and for some norm.
So when the resolution is doubled, the error becomes $Error=C(\Delta x)^2/4$ et cetera.
{\em This means that the error should in principle go down by a factor of four.
So for $\Delta x$ smaller and smaller this convergence should be obeyed and observed by you.}
If this is not the case something is wrong: your time step was not sufficiently small
and one is measuring the temporal error instead, there is still a bug, the measurement time $t_c$ is
inappropriate, or the boundary conditions are implemented with another order, or something else to be
determined by you. 
\item It is therefore easiest to use regular refinements, e.g., with mesh sizes of
$\Delta x, \Delta x/2, \Delta x/4, \Delta x/8$;
but do take care that when a high resolution solution is used as pseudo-exact solution,
its mesh size is sufficiently smaller than the smallest mesh size of the numerical test solutions.
\item When the order of the scheme is unknown one can assume $Error=C(\Delta x)^{\gamma}$ at first and determine the exponent $\gamma$,
using logarithmic plots.
\item Graphs should have labels of appropriate size and should be readable.
\item As requested and required for your own sake, the mathematics of the algorithm should be worked out on paper first in detail.
Code is not acceptable as explanation of the algorithm.
The indexing of the grid  should be clearly indicated.
The treatment and adaptation of the main algorithm at the boundary points should be clearly explained
in terms of mathematical formulas.
\item There are alternative ways of considering the convergence without having an exact or fine numerical solution.
Assume one has the numerical solutions at four resolutions $h=\Delta x,\Delta x/2,\Delta x/4,\Delta x/8$.
Denote these by $u_h,u_{h/2},u_{h/4}, u_{h/8}$. Assume an error solution Ansatz of the form $C(\Delta x)^2$
Consider the ratio
\begin{align}
\frac {h^2-(h/2)^2}{(h/4)^2-(h/8)^2} = \frac {(||u_{h/8}-u_{h}||-||u_{h/8}-u_{h/2}||)}{(||u_{h/8}-u_{h/2}||-||u_{h/8}-u_{h/4}||)}
\end{align}
for some norm $||\cdot||$ and analyse this using the error Ansatz. This is called Richardson's extrapolation.
Do the same for an error Ansatz with unknown exponent $\gamma$.
\item Something similar can be done to verify convergence in time, using refinements for of the time step $\Delta t$.
\item It is advisable to make a routine such that one can plot the solutions
at desired times specified before the time loop.
\item Mention your name in the (Matlab) codes you submit. Also submit the codes used to calculate the tables of convergence
and the plotting. These should be part of your programs.
\end{itemize}


\end{document}
  
\newpage
\centerline{\bf References}


\noindent
Internet.

\noindent



\begin{figure}
\centerline{
\includegraphics[width=0.5\textwidth,height=0.4\textwidth]{volcanicdike3.pdf}
}
\caption{Photopgraph of the groundwater cell in Wetropolis and sketch of the set-up with the variables defined.}
\LF{sketgrw}
\end{figure}

\centerline{\em Onno Bokhove, October 2016.}

\end{document}

